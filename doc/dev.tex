\documentclass[11pt,a4paper]{article}
\usepackage{amsmath,amsfonts,amsthm,amssymb}
\usepackage[colorlinks,linkcolor=blue]{hyperref}
\usepackage{graphicx,float,wrapfig,subfig}
\usepackage{algorithmic}
\usepackage{xeCJK}
\usepackage{fontspec}
\usepackage{enumerate}
\usepackage{setspace}
\usepackage{indentfirst}

% Chinese support
\setCJKmainfont[BoldFont={Adobe Heiti Std}, ItalicFont={Adobe Kaiti Std}]{Adobe Song Std}
\setCJKsansfont{WenQuanYi Zen Hei}
\setCJKmonofont{Adobe Fangsong Std}
\punctstyle{quanjiao}

% Spacing and margins
\onehalfspacing
\allowdisplaybreaks
\parindent=2em
\parskip=1ex
\addtolength{\hoffset}{-2em}
\addtolength{\voffset}{-10ex}
\addtolength{\textwidth}{4em}
\addtolength{\textheight}{5ex}


\begin{document}

% Make title
\title{\textbf{第16届智能体大赛AI开发手册}}
\author{\texttt{清华大学计算机系科协}}
\date{\today}
\maketitle
\thispagestyle{empty}

% Homework begin from here
\section{快速上手}
\begin{enumerate}[(a)]
\item 将/ai/sampleAI文件夹复制一份,改成自己喜欢的名字,比如/ai/bestAI。
\item 运行/ai/bestAI/compile.bat。
\item 运行/run.bat,即可享受对战的快感。
\end{enumerate}

\section{基础说明}
\begin{enumerate}[(a)]
\item /ai/myAI是一个空的AI框架,/ai/sampleAI是填充好的一个简单AI。里面有详尽的注释,读一读代码即可搞清楚如何编写AI。
\item 运行gui以后,选择AI对战的方法是:按下鼠标选择Boss的AI,移动鼠标到Plane的AI处后松开。
\item 对战开始以后会弹出3个黑框,其中platform.exe会不停输出两个数字,即回合数和积分。如果没有出现,则关闭所有窗口重新开启即可。
\item 观看录像时,因为gui无法精确模拟连续时间的游戏情况,所以会导致内核判定子弹击中Plane,但是显示出来有几像素的偏差这种问题。以内核判定为准。
\end{enumerate}

\textbf{以上就是写一个AI所需要了解的所有基本情况。下面将会介绍一些游戏内核相关的内容,第一次阅读本文档可以忽略。}

\section{深入理解}
\subsection{游戏即时性的实现方式}
游戏采取socket通信的形式,一个server负责处理游戏逻辑,两个client分别作为Boss和Plane。在游戏正常进行到第i回合时,server端做的事情如下:
\begin{enumerate}[(a)]
\item 将第i回合开始时的游戏局面信息传递给client。
\item 等待0.1s。
\item 处理client新发送过来的动作,动作的起始时间要大于等于i。
\item 判断i到i+1回合内Plane有没有被击中。
\item 更新i+1回合的游戏信息。
\item 生成第i回合的录像信息。
\end{enumerate}

在client端,有一个线程来接收server发送过来的游戏信息,整个client端可以用伪代码来表示其运行情况:
\begin{algorithmic}
\WHILE{游戏没有结束}
	\IF{没有收到新的游戏信息}
		\STATE continue
	\ELSE
		\STATE 调用ai的dll获得玩家指定的动作
		\STATE 发送动作到server
	\ENDIF
\ENDWHILE
\end{algorithmic}

因此,client端可以计算任意长的时间,只要保证动作发送到server时,游戏的回合数小于等于动作的起始时间即可。

\subsection{Plane被子弹击中的判定方法}
Plane是一个点,子弹是一个圆。在任意时刻,只要Plane和子弹中心的距离小于等于子弹半径即被判定为击中。这里需要用到一些初、高中数学方面的知识。

假设某一时刻,某子弹位置是$(x_b,y_b)$,速度为$(u_x,u_y)$,半径为$r$;飞机位置是$(x_p,y_p)$,速度为$(v_x,v_y)$。那么经过$t\times 0.1s$后子弹位于$(x_b+u_xt,y_b+u_yt)$,飞机位于$(x_p+v_xt,y_p+v_yt)$。
二者Euclid距离与子弹半径的平方差$$d(t)=(x_b+u_xt-x_p-v_xt)^2+(y_b+u_yt-y_p-v_yt)^2-r^2=At^2+Bt+C$$
其中
\begin{align*}
A&=(u_x-v_x)^2+(u_y-v_y)^2\\
B&=2\left[(u_x-v_x)(x_b-x_p)+(u_y-v_y)(y_b-y_p)\right]\\
C&=(x_b-x_p)^2+(y_b-y_p)^2-r^2
\end{align*}
因此,二者在接下来的0.1s内会相撞的充要条件是$$\exists t \in [0,1],~s.t.~d(t) \leq 0$$
所以我们只要求出$d(t)$在$[0,1]$的最小值,判断其是否小于等于0即可。
\end{document}
